%%%%%%%%%%%%%%%%%%%%%%%%%%%%%
%               %
%%%%%%%%%%%%%%%%%%%%%%%%%%%%%
\documentclass[12pt]{article}

% colors
\usepackage[dvipsnames,svgnames,x11names]{xcolor}

% graphics
\usepackage{graphics}

% TikZ picture environment
\usepackage{tikz}
\usetikzlibrary{arrows,scopes,fadings}

% Math formatting
\usepackage{amsmath}
% More math
\usepackage[fleqn,tbtags]{mathtools}
% Physics symbols
\usepackage{physics}

%\usepackage{mdwlist}

% tables
\usepackage{tabu}
\usepackage{diagbox}

%Page Margins
\usepackage{geometry}
\geometry{top=1.5in,left=1in, right=1in, bottom = 1.5in}


% Useful commands
	% Answer in write up
	\newcommand{\ans}[1]{\textbf{\color{Black} #1}}
	% Write up on the board
	%\newcommand{\board}[1]{\textbf{\color{Red} #1}}
	% ask question
	%\newenvironment{question}{\begin{quotation} \noindent \textbf{\color{Blue} Question}\:\:}{\end{quotation}}
	%\newcommand{\ask}[2]{\begin{quotation} \noindent {\color{Blue} \textbf{Question:}\:\: #1} \\ {\color{Purple} #2}\end{quotation}}
	% answer
	%\newcommand{\ans}[1]{{\color{Purple} #1}}
	
% Vectors
	\renewcommand{\vec}[1]{\boldsymbol{#1}}
	\newcommand{\unitvec}[1]{\boldsymbol{\hat{#1}}}

\begin{document}

\title{Relative Error $L_{2}$ Table: Classical Method for Inviscid Burgers}
\author{}
\date{}

\maketitle



\subsection{Inviscid Burgers Equation}

\begin{itemize}
    \item The relative error $L_{2}$ with initial condition $u_{0}(x) = -\sin(\pi \frac{x}{8})$
\end{itemize}

\centering
\begin{tabular}{|l|c|c|c|c|   }\hline
\diagbox[width=9em]{Time\\step\\$\Delta t$}{\\Relative\\Error $L_{2}$ at\\$T$ }&
  $T=2.00$ & $T=1.99$ & $T=1.0$ & $T=0.5$ \\ \hline
 $0.064$ & $9.501 \times 10^{-2}$ & $9.346 \times 10^{-2}$ & $4.407 \times 10^{-2}$ & $2.590 \times 10^{-2}$ \\ \hline
 $0.048$ & $9.647 \times 10^{-2}$ & $9.488 \times 10^{-2}$ & $4.248 \times 10^{-2}$ & $2.456 \times 10^{-2}$ \\ \hline
 $0.032$ & $9.548 \times 10^{-2}$ & $9.393 \times 10^{-2}$ & $4.290 \times 10^{-2}$ & $2.453 \times 10^{-2}$ \\ \hline
\end{tabular}



\begin{itemize}
    \item The relative error $L_{2}$ with initial condition $u_{0}(x) = \cos(-\pi \frac{x}{8})$
\end{itemize}

\centering
\begin{tabular}{|l|c|c|c|c|   }\hline
\diagbox[width=9em]{Time\\step\\$\Delta t$}{\\Relative\\Error $L_{2}$ at\\$T$ }&
  $T=2.00$ & $T=1.99$ & $T=1.0$ & $T=0.5$ \\ \hline
 $0.064$ & $8.988 \times 10^{-2}$ & $8.806 \times 10^{-2}$ & $2.978 \times 10^{-2}$ & $1.613 \times 10^{-2}$ \\ \hline
 $0.048$ & $9.148 \times 10^{-2}$ & $8.964 \times 10^{-2}$ & $2.665 \times 10^{-2}$ & $1.296 \times 10^{-2}$ \\ \hline
 $0.032$ & $9.032 \times 10^{-2}$ & $8.849 \times 10^{-2}$ & $2.759 \times 10^{-2}$ & $1.305 \times 10^{-2}$ \\ \hline
\end{tabular}


\begin{itemize}
    \item Remark: (from course notes: Lax Convergence Theorem) the actual error $E^{h}$ is converging with the same order as the local truncation error $T^{h}$ i.e the order of convergence is $q=2$ in space-time. 
\end{itemize}


\end{document}
